\documentclass{article}
\usepackage[margin=2cm]{geometry}
\usepackage{enumitem}
\usepackage{amsmath}

\setlist[itemize]{noitemsep} 
\setlist[enumerate]{noitemsep} 

\begin{document}

\title{D\&D 5e Wargame Experiment}
\author{Matt}
\maketitle

\section{Idea}

The principle properties the wargame rules should satisfy are the following:
\begin{enumerate}
    \item \label{itm:smooth} Gameplay should be relatively simple and smooth;
        i.e., no complex calculations during play.
    \item \label{itm:auto} Rules should be automatically derived from the situation,
        without the need for ad-hoc rule introductions for each situation in play.
    \item \label{itm:strength} Creature strength should be true to the stats of the individual creatures.
    \item \label{itm:scale} Rules should scale from smallish skirmishes to large battles.
\end{enumerate}

In order to address Goal~\ref{itm:smooth}, we separate the conversion into two parts:
the complex rules to convert a D\&D situation into a wargame situation (Section~\ref{sec:pregame}),
and the simpler rules to play the wargame (Section~\ref{sec:game}).
The conversion rules are pure mathematical calculations from creature statistics,
satisfying Goal~\ref{itm:auto} and Goal~\ref{itm:strength}.
Finally, we use parameters to scale statistics to an appropriate size for the situation,
satisfying Goal~\ref{itm:scale}.

\section{Pre-gameplay}\label{sec:pregame}

Before a battle starts,
calculations are performed based on the participants' D\&D statistics.

\subsection{Parameters}

The parameters are used to scale the battle to an appropriate size.
Lower values correspond to greater granularity
and will result in slower, less eventful battles of attrition.
Higher values correspond to greater abstraction
and will reult in quick battles where initiative has an important impact.

\begin{itemize}
    \item $A$ (attack scalar) -- roughly indicates how much damage is required to reduce a unit's health by one point
    \item $H$ (health scalar) -- indicates how many points of creature health (without armor) are equal to one point of unit health
\end{itemize}

\subsection{Variables}

We take a single creature's base statistics along with the number of creatures in a unit
in order to calculate that unit's overall strength.

\subsubsection{Input}

As input, we use a creature's health and AC for its defense,
and the damage and attack bonus for each of its attacks,
as well as the unit size.

\begin{itemize}
    \item $h$ (creature health) -- the maximum hitpoints of a single creature
    \item $r$ (creature AC) -- the AC of a creature
    \item $d_a$ (attack damage) -- the average damage done by attack $a$
    \item $b_a$ (attack bonus) -- the to-hit bonus of attack $a$
    \item $u$ (unit size) -- the number of creatures in a unit
\end{itemize}

\subsubsection{Output}

There are two output values.
The number of dice is an aggregation of a unit's defensive strength and number.
It is used as a set of hitpoints during combat.
The size of the die is an aggregation
of a unit's offensive strength and accuracy \emph{per die}.
Regardless of unit size,
two units of the same type will always use the same die size
(but may use a different number of dice).
Smaller die size is stronger,
as the goal is to roll a 1,
resulting in reducing the opponent's number of dice by 1.

\begin{itemize}
    \item $n$ (number of dice) -- the number of dice to use for attacks
    \item $s_a$ (die size) -- the size of the die to use for attack $a$
\end{itemize}

\subsection{Calculations}

The $n$ and $s$ values are calculated according to the following formulas:

\begin{align*}
    n   &=  \left\lceil\frac{hu}{H}\right\rceil \\
    s_a &=  \left\lceil\frac{A}{d_a(\frac{H}{h})}\right\rceil
\end{align*}

The $n$ formula is explained simply:
It is the total health of the unit divided by the $H$ parameter,
rounded up.

The $s_a$ formula is more complex,
as it must take into account both attacking strength and defensive strength.
Because $n$ dice (proportional to defense) will be rolled,
in order to avoid a squared impact of defense,
the defense must be divided out of the equation.

The development of the formula is as follows:
The attack value \emph{per creature} is proportional to 
\[
    d_a
\]
The total attack value of the unit is then 
\[
    d_a u
\]
We divide this by the number of dice to get attack value per die: 
\[
    \frac{d_a u}{n}
\]
But to reduce rounding error we instead use the unrounded formula for $n$: 
\[
    \frac{d_a u}{\frac{hu}{H}}
\]
which simplifies to 
\[
    d_a\left(\frac{H}{h}\right)
\]
This value is proportional to our hit chance.
We scale it to a probability by dividing by $A$:
\[
    \frac{d_a(\frac{H}{h})}{A}
\]
We now have our true hit chance.
To turn this into a die, we take the reciprocal and round up.
\[
    \left\lceil\frac{A}{d_a(\frac{H}{h})}\right\rceil
\]

\subsection{Examples}

We now demonstrate the use of the formulas through examples.

\subsubsection{Unit of 20 Orcs}

An orc has the following relevant statistics:
\begin{itemize}
    \item $h = 15$ HP (2d8 + 6)
    \item $r = 13$ AC
    \item Javelin Attack:
        \begin{itemize}
            \item $a_j = +5$ to hit
            \item $d_j = 6.5$ damage (d6 + 3)
        \end{itemize}
    \item Greataxe Attack:
        \begin{itemize}
            \item $a_g = +5$ to hit
            \item $d_g = 9.5$ damage (d12 + 3)
        \end{itemize}
\end{itemize}

For 20 orcs, given $A = 500$ and $H = 100$
\begin{alignat*}{2}
    n   =&  \left\lceil\frac{(15)(20)}{100}\right\rceil             =&  3   \\
    s_j =&  \left\lceil\frac{500}{6.5(\frac{100}{15})}\right\rceil  =&  12  \\
    s_g =&  \left\lceil\frac{500}{9.5(\frac{100}{15})}\right\rceil  =&  8
\end{alignat*}


\section{Gameplay}\label{sec:game}

\section{Choosing Parameters}\label{sec:params}

Parameter selection is important for smooth and interesting gameplay.
One may choose from a number of strategies to help decide on parameters.

\subsection{Max Hit Chance}

One may select the maximum hit chance that they would like to see in the game.
For example, one may decide that the strongest unit should always hit (i.e., roll a d1)
or have a 50\% chance of hitting (i.e., roll a d2).
Taking all units into account and a given $H$ value,
an $A$ value is selected such that the strongest unit will roll the selected die.

We now effectively have a new parameter $S$, representing the die of the strongest attack.
We can then reverse the die formula to derive a formula for $A$:
\[
    S = \left\lceil\frac{A}{d_a(\frac{H}{h})}\right\rceil
\]
We remove rounding and get
\[
    S = \frac{A}{d_a(\frac{H}{h})}
\]
which we can rearrange to get
\[
    A = S(d_a)\left(\frac{H}{h}\right)
\]

This strategy guarantees a certain hit chance and clearly delineates a strongest unit.
It must be noted, however, that if a small die size $S$ is chosen,
then the unit with the second-strongest attack will be at an otherwise great disadvantage,
as their die size will be rounded to at least $n + 1$.


\subsubsection{Example: Orcs Roll d2}

We consider again the 20 orcs example with $H = 100$.
We want them to have a 50\% chance to hit.
We solve:
\begin{alignat*}{2}
    A &= 2(9.5)\left(\frac{100}{15}\right) &{}\approx{} 126.67
\end{alignat*}

\end{document}