\documentclass[twocolumn]{article}
\usepackage[margin=2cm]{geometry}
\usepackage{enumitem}
\usepackage{amsmath}
\usepackage{todonotes}

\setlist[itemize]{noitemsep} 
\setlist[enumerate]{noitemsep} 
\setlength{\marginparwidth}{2cm}

\begin{document}

\title{D\&D 5e Wargame Experiment}
\author{Matt}
\maketitle

\section{Idea}

The principle properties the wargame rules should satisfy are the following:
\begin{enumerate}
    \item \label{itm:smooth} Gameplay should be simple and smooth;
        i.e., no calculations and few statistics to lookup during play.
    \item \label{itm:auto} Rules should be automatically derived from the situation,
        without the need for ad-hoc rule introductions for each situation in play.
    \item \label{itm:strength} Creature strength should be true to the stats of the individual creatures.
    \item \label{itm:scale} Rules should scale from smallish skirmishes to large battles.
\end{enumerate}

In order to address Goal~\ref{itm:smooth}, we separate the conversion into two parts:
the complex rules to convert a D\&D situation into a wargame situation (Section~\ref{sec:pregame}),
and the simpler rules to play the wargame (Section~\ref{sec:game}).
The conversion rules are pure mathematical calculations from creature statistics,
satisfying Goal~\ref{itm:auto} and Goal~\ref{itm:strength}.
Finally, we use parameters to scale statistics to an appropriate size for the situation,
satisfying Goal~\ref{itm:scale}.

The result is a system where each unit in the battle boils down to two numbers.
Very little math has to be performed during play,
to permit management of large battlefields
with minimal confusion.

\section{Pre-gameplay}\label{sec:pregame}

Before a battle starts,
calculations are performed based on the participants' D\&D statistics.

\subsection{Parameters}

The parameters are used to scale the battle to an appropriate size.
They control the likelihood of an attack to hit,
and number of dice rolled,
in turn affecting the duration of the battle.

\begin{itemize}
    \item $A$ (attack scalar) -- roughly indicates how much damage is required to reduce a unit's health by one point
    \item $H$ (health scalar) -- indicates how many points of creature health (without armor) are equal to one point of unit health
    \item $P$ (accuracy exponent) -- indicates the compounding force of accuracy
\end{itemize}

Provided they are ascribed valid values,
the invariant is maintained that if army $A$ is objectively stronger than army $B$,
the resulting statistics of army $A$ will be at least as strong as those of army $B$.

The valid ranges for the parameters are the following:
\begin{align*}
    A > 0 \\
    H > 0 \\
    P \geq 0
\end{align*}
Section~\ref{sec:params} contains further guidance on selecting values for the parameters.

\subsection{Variables}

We take a single creature's base statistics along with the number of creatures in a unit
in order to calculate that unit's overall strength.

\subsubsection{Input}

As input, we use a creature's health and AC for its defense,
and the damage and attack bonus for each of its attacks,
as well as the unit size.

\begin{itemize}
    \item $h$ (creature health) -- the maximum hitpoints of a single creature
    \item $r$ (creature AC) -- the AC of a creature
    \item $d_a$ (attack damage) -- the average damage done by attack $a$
    \item $b_a$ (attack bonus) -- the to-hit bonus of attack $a$
    \item $u$ (unit size) -- the number of creatures in a unit
\end{itemize}

\subsubsection{Output}

There are two output values.
The number of dice is an aggregation of a unit's defensive strength and number.
It is used as a set of hitpoints during combat.
The size of the die is an aggregation
of a unit's offensive strength and accuracy \emph{per die}.
Regardless of unit size,
two units of the same type will always use the same die size
(but may use a different number of dice).
Smaller die size is stronger,
as the goal is to roll a 1,
resulting in reducing the opponent's number of dice by 1.

\begin{itemize}
    \item $n$ (number of dice) -- the number of dice to use for attacks
    \item $s_a$ (die size) -- the size of the die to use for attack $a$
\end{itemize}

\subsection{Calculations}

The $n$ and $s$ values are calculated according to the following formulas:

\begin{align*}
    n   &=  
        \left\lceil
            \frac
                {h u r^P}
                {10^P H}
        \right\rceil \\
    s_a &=
        \left\lceil
            \frac
                {h r^P A}
                {d_a (10 + b_{a})^P H}
        \right\rceil
\end{align*}

\subsubsection{Accuracy}

D\&D's concept of accuracy is a source of complications in the formulas,
and must be addressed specifically.

We begin by observing that in standard D\&D rules,
an increase by a factor of $x$ in an attack's accuracy
can be roughly equivalently represented by a decrease in a defender's
health by a factor of $x$.
For example, if an attack normally has a 25\% chance of hitting a defender with 2 hitpoints,
we might reasonably expect the defender to last 8 rounds.\footnote{
    This approximation follows intuition,
    but is not statistically correct with respect to expected values.
    It is considered sufficient for this draft
    but may be revisited in a future version.
}
If we raise the accuracy to 50\%,
the defender would need around 4 hitpoints to last 8 rounds.

Accuracy of attacks in standard D\&D is not derived from one statistic,
but from two: the attacker's attack bonus and the defender's AC.
Modelling this directly would present a problem for the transformation,
as it would necessitate an additional statistic and comparison during gameplay.
Instead, we try to bundle the effect of AC the $n$,
and the effect of attack bonus into $s_a$.

It is not possible to replicate the effect of accuracy fully, however.
To demonstrate, we turn to an example of a commonner (+2 attack bonus)
attacking a tarrasque (25 AC).
Barring critical hits, the commoner will never be able to hit the tarrasque,
as the maximum roll of 22 is less than the tarrasque's AC.
The result is that the $n$ value for a tarrasque unit must be infinite,
or that the $s_a$ value for a commoner unit must be infinite.

Clearly, neither of these options is practical.
Instead of modelling the asymptotic effect of AC and attack bonus directly,
we use power function to approximate it.
AC is converted into a defence factor:
\[
    \left(\frac{r}{10}\right)^P
\]
AC is first divided by 10,
as 10 is the standard AC without bonuses or penalties,
then raised to a power $P$ which represents the multiplicative power of AC.

The factor for attack bonus is derived similarly,
but 10 added, establishing 0 as the baseline attack bonus:
\[
    \left(\frac{10 + b_a}{10}\right)^P
\]

\subsubsection{Deriving the Formulas}

The $n$ formula is relatively simple:
It is the total health of the unit,
multiplied by a factor representing the value of armor,
divided by the $H$ parameter:
\[
    \frac
        {h u (\frac{r}{10})^P}
        {H}
\]
We round and simplify to get our value for $n$.

The $s_a$ formula is more complex,
as it must take into account both attacking strength and defensive strength.
Because $n$ dice (proportional to defense) will be rolled,
in order to avoid a compounding impact of defense,
the defense must be divided out of the equation.

The development of the formula is as follows:
The attack value \emph{per creature} is proportional to
the average damage multiplied by the attack bonus factor.
\[
    d_a \left(\frac{10 + b_{a}}{10}\right)^{P}
\]
The total attack value of the unit is then 
\[
    d_a u \left(\frac{10 + b_{a}}{10}\right)^{P}
\]
We divide this by the number of dice to get attack value per die: 
\[
    \frac
        {d_a u \left(\frac{10 + b_{a}}{10}\right)^{P}}
        {n}
\]
But to reduce rounding error we instead use the unrounded formula for $n$: 
\[
    \frac
        {d_a u \left(\frac{10 + b_{a}}{10}\right)^{P}}
        {\frac{h u r^P}{10^P H}}
\]
which simplifies to 
\[
    \frac
        {d_a (10 + b_{a})^P H}
        {h r^P}
\]
This value is proportional to our hit chance.
We scale it to a probability by dividing by $A$:
\[
    \frac
        {d_a (10 + b_{a})^P H}
        {h r^P A}
\]
We now have our true hit chance.
To turn this into a die, we take the reciprocal and round up.
\[
    \left\lceil
        \frac
            {h r^P A}
            {d_a (10 + b_{a})^P H}
    \right\rceil
\]

\subsection{Examples}

We now demonstrate the use of the formulas through examples.

\subsubsection{Unit of 20 Orcs}

An orc has the following relevant statistics:
\begin{itemize}
    \item $h = 15$ HP (2d8 + 6)
    \item $r = 13$ AC
    \item Javelin Attack:
        \begin{itemize}
            \item $b_j = +5$ to hit
            \item $d_j = 6.5$ damage (d6 + 3)
        \end{itemize}
    \item Greataxe Attack:
        \begin{itemize}
            \item $b_g = +5$ to hit
            \item $d_g = 9.5$ damage (d12 + 3)
        \end{itemize}
\end{itemize}

For 20 orcs, given $A = 500$, $H = 100$, and $P = 2$
\begin{alignat*}{2}
    n   =&  
        \left\lceil
            \frac
                {(15) (20) (13^2)}
                {(10^2) (100)}
        \right\rceil
            &=  6
    \\
    s_a =&
        \left\lceil
            \frac
                {(15) (13^2) (500)}
                {(6.5) (10 + 5)^2 (100)}
        \right\rceil
            &=   26 \\
    s_g =&  
        \left\lceil
            \frac
                {(15) (13^2) (500)}
                {(9.5) (10 + 5)^2 (100)}
        \right\rceil
            &=  18
\end{alignat*}

\subsubsection{Armored vs Unarmored Commoners}

Here we see the effect of armor on the statistics.
A commoner has the following relevant statistics:

\begin{itemize}
    \item $h = 4.5$ HP (1d8)
    \item $r = 10$ AC
    \item Club Attack:
        \begin{itemize}
            \item $b_c = +2$ to hit
            \item $d_c = 2.5$ damage (1d4)
        \end{itemize}
\end{itemize}


For 100 commoners, given $A = 500$, $H = 100$, and $P = 2$

\begin{alignat*}{2}
    n   &=  
        \left\lceil
            \frac
                {(4.5) (100) (10^2)}
                {(10^2) (100)}
        \right\rceil
            &= 5
    \\
    s_c &=
        \left\lceil
            \frac
                {(4.5) (10^2) (500)}
                {(2.5) (10 + 2)^P (100)}
        \right\rceil
            &= 7
\end{alignat*}

Equipped with chain shirts, the commoners' AC rises to 13.
This affects their statistics as follows:

\begin{alignat*}{2}
    n   &=  
        \left\lceil
            \frac
                {(4.5) (100) (13^2)}
                {(10^2) (100)}
        \right\rceil
            &= 8
    \\
    s_c &=
        \left\lceil
            \frac
                {(4.5) (13^2) (500)}
                {(2.5) (10 + 2)^P (100)}
        \right\rceil
            &= 11
\end{alignat*}

Note that while the armored commoners have more dice,
representing their superior defense,
their larger die size causes each individual die less likely to land a hit.
As the units have the same size and weapons,
their ought to have the same attacking power.
Confirming the validity of the conversion,
the expected damage of each unit is the same,
within a rounding error:
$\frac{5}{7} \approx 71\%$ for the unarmored commoners,
and $\frac{8}{11} \approx 73\%$ for the armored commoners.

\section{Gameplay}\label{sec:game}
\missingfigure{diagram of play}

\section{Choosing Parameters}\label{sec:params}

Parameter selection is important for smooth and interesting gameplay.
One should consider a number of options when deciding on parameter values.

\subsection{The $P$ parameter}

The $P$ parameter is the most nebulous of the parameters,
as it doesn't correspond to any real statistic.
The current recommendation is to assign it a value of 2.

\subsection{Max Hit Chance}

One may select the maximum hit chance that they would like to see in the game.
For example, one may decide that the strongest unit should always hit (i.e., roll a d1)
or have a 50\% chance of hitting (i.e., roll a d2).
Taking all units into account and a given $H$ value and $P$ value,
an $A$ value is selected such that the strongest unit will roll the selected die.

We now effectively have a new parameter $S$, representing the die of the strongest attack.
We can then reverse the die formula to derive a formula for $A$:
\[
    S =
        \left\lceil
            \frac
                {h r^P A}
                {d_a (10 + b_{a})^P H}
        \right\rceil
\]
After removing rounding, we can rearrange the formula to derive $A$:
\[
    A = 
        \frac
            {d_a (10 + b_{a})^P H S}
            {h r^P}
\]

This strategy guarantees a certain hit chance and clearly delineates a strongest unit.
It must be noted, however, that if a small die size $S$ is chosen,
then the unit with the second-strongest attack will be at an otherwise great disadvantage,
as their die size will be rounded to at least $n + 1$.


\subsubsection{Example: Orcs Roll d2}

We consider again the 20 orcs example with $H = 100$ and $P = 2$.
We want them to have a 50\% chance to hit.
We solve:
\begin{alignat*}{2}
    A &= 
        \frac
            {(9.5) (10 + 5)^2 (100) (2)}
            {(15) (13)^2}
        &{}\approx{} 168.63
\end{alignat*}

\end{document}