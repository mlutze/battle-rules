\documentclass[twocolumn]{article}
\usepackage[margin=2cm]{geometry}
\usepackage{enumitem}
\usepackage{amsmath}
\usepackage{todonotes}
\usepackage{units}

\setlist[itemize]{noitemsep} 
\setlist[enumerate]{noitemsep} 
\setlength{\marginparwidth}{2cm}

\begin{document}

\title{D\&D 5e Wargame Experiment}
\author{Matt}
\maketitle

\section{Idea}

The principle properties the wargame rules should satisfy are the following:
\begin{enumerate}
    \item \label{itm:smooth} Gameplay should be simple and smooth;
        i.e., no calculations and few statistics to lookup during play.
    \item \label{itm:auto} Rules should be automatically derived from the situation,
        without the need for ad-hoc rule introductions for each situation in play.
    \item \label{itm:strength} Creature strength should be true to the stats of the individual creatures.
    \item \label{itm:scale} Rules should scale from smallish skirmishes to large battles.
\end{enumerate}

In order to address Goal~\ref{itm:smooth}, we separate the conversion into two parts:
the complex rules to convert a D\&D situation into a wargame situation (Section~\ref{sec:pregame}),
and the simpler rules to play the wargame (Section~\ref{sec:game}).
The conversion rules are pure mathematical calculations from creature statistics,
satisfying Goal~\ref{itm:auto} and Goal~\ref{itm:strength}.
Finally, we use parameters to scale statistics to an appropriate size for the situation,
satisfying Goal~\ref{itm:scale}.

The result is a system where each unit in the battle boils down to two numbers.
Very little math has to be performed during play,
to permit management of large battlefields
with minimal confusion.

\section{Pre-gameplay}\label{sec:pregame}

Before a battle starts,
calculations are performed based on the participants' D\&D statistics.

\subsection{Parameters}

The parameters are used to scale the battle to an appropriate size.
They control the likelihood of an attack to hit,
and number of dice rolled,
in turn affecting the duration of the battle.

\begin{itemize}
    \item $A$ (attack scalar) -- roughly indicates how much damage is required to reduce a unit's health by one point
    \item $H$ (health scalar) -- indicates how many points of creature health (without armor) are equal to one point of unit health
    \item $P$ (accuracy exponent) -- indicates the compounding force of accuracy
    \item $S$ (the size of the die) -- the die to be rolled to determine if attacks hit
\end{itemize}

Provided they are ascribed valid values,
the invariant is maintained that if army $A$ is objectively stronger than army $B$,
the resulting statistics of army $A$ will be at least as strong as those of army $B$.

The valid ranges for the parameters are the following:
\begin{align*}
    A > 0 \\
    H > 0 \\
    P \geq 0 \\
    S > 0
\end{align*}
Section~\ref{sec:params} contains further guidance on selecting values for the parameters.

\subsection{Variables}

We take a single creature's base statistics along with the number of creatures in a unit
in order to calculate that unit's overall strength.

\subsubsection{Input}

As input, we use a creature's health and AC for its defense,
and the damage and attack bonus for each of its attacks,
as well as the unit size.

\begin{itemize}
    \item $h$ (creature health) -- the maximum hitpoints of a single creature
    \item $r$ (creature AC) -- the AC of a creature
    \item $d_a$ (attack damage) -- the average damage done by attack $a$
    \item $b_a$ (attack bonus) -- the to-hit bonus of attack $a$
    \item $u$ (unit size) -- the number of creatures in a unit
    \item $c$ (creature size) -- the length of a tile that a creature occupies
        (e.g., 5 feet for a medium creature)
    \item $p$ (creature speed) -- the movement speed of a creature
\end{itemize}

\subsubsection{Output}

There are two output values.
The number of dice is an aggregation of a unit's defensive strength and number.
It is used as a set of hitpoints during combat.
The roll target (or \emph{hurdle})\footnote{
    The term \emph{hurdle} is used to emphasize
    that the roll must be \emph{above} the given value to succeed,
    rather than meeting it.
}
is an aggregation
of a unit's offensive strength and accuracy \emph{per die}.
Regardless of unit size,
two units of creatures with the same statistics
will always have the same roll target
(but may use a different number of dice).
A lower roll target is stronger,
as the goal is to roll above the target,
resulting in reducing the opponent's number of dice by 1.

\todo[inline]{header for per-creature values}
\begin{itemize}
    \item $n$ (number of dice) -- the number of dice to use for attacks
    \item $t_a$ (target/hurdle) -- the number to roll above in order to succeed with attack $a$
    \item $m$ (movement speed) -- the number of tiles a unit moves per turn
    \item $s$ (seconds per turn) -- the length of one turn of gameplay in seconds
    \item $e$ (edge length) -- the length of the edge of a grid tile in feet\footnote{
        The values $s$ and $e$ are global
    }
\end{itemize}

\todo[inline]{header for global values}
\begin{itemize}
    \item $s$ (seconds per turn) -- the length of one turn of gameplay in seconds
    \item $e$ (edge length) -- the length of the edge of a grid tile in feet
\end{itemize}

\todo[inline]{explain global/per-creature distinction}

The combination of the two statistics is written $\nicefrac{n}{t_a}$
and read ``$n$ over $t_a$''.

\subsection{Calculations}

The $n$ and $t_a$ values are calculated according to the following formulas:

\begin{align*}
    n   &=  
        \left\lfloor
            \frac
                {h u r^P}
                {10^P H}
        \right\rceil \\
    t_a &=
        20 -
        \left\lfloor
            \frac
                {d_a (10 + b_{a})^P H S}
                {h r^P A}
        \right\rceil
\end{align*}

\subsubsection{Accuracy}

D\&D's concept of accuracy is a source of complications in the formulas,
and must be addressed specifically.

We begin by observing that in standard D\&D rules,
an increase by a factor of $x$ in an attack's accuracy
can be roughly equivalently represented by a decrease in a defender's
health by a factor of $x$.
For example, if an attack normally has a 25\% chance of hitting a defender with 2 hitpoints,
we might reasonably expect the defender to last 8 rounds.\footnote{
    This approximation follows intuition,
    but is not statistically correct with respect to expected values.
    It is considered sufficient for this draft
    but may be revisited in a future version.
}
If we raise the accuracy to 50\%,
the defender would need around 4 hitpoints to last 8 rounds.

Accuracy of attacks in standard D\&D is not derived from one statistic,
but from two: the attacker's attack bonus and the defender's AC.
Modeling this directly would present a problem for the transformation,
as it would necessitate an additional statistic and comparison during gameplay.
Instead, we try to bundle the effect of AC the $n$,
and the effect of attack bonus into $t_a$.

It is not possible to replicate the effect of accuracy fully, however.
To demonstrate, we turn to an example of a commoner (+2 attack bonus)
attacking a tarrasque (25 AC).
Barring critical hits, the commoner will never be able to hit the tarrasque,
as the maximum roll of 22 is less than the tarrasque's AC.
The result is that the $n$ value for a tarrasque unit must be infinite,
or that the $t_a$ value for a commoner unit must be at least 20.

Clearly, neither of these options is practical.
Instead of modeling the asymptotic effect of AC and attack bonus directly,
we use power function to approximate it.
AC is converted into a defense factor:
\[
    \left(\frac{r}{10}\right)^P
\]
AC is first divided by 10,
as 10 is the standard AC without bonuses or penalties,
then raised to a power $P$ which represents the multiplicative power of AC.

The factor for attack bonus is derived similarly,
but 10 added, establishing 0 as the baseline attack bonus:
\[
    \left(\frac{10 + b_a}{10}\right)^P
\]

\subsubsection{Deriving the Formulas}

The $n$ formula is relatively simple:
It is the total health of the unit,
multiplied by a factor representing the value of armor,
divided by the $H$ parameter:
\[
    \frac
        {h u (\frac{r}{10})^P}
        {H}
\]
We round and simplify to get our value for $n$.

The $t$ formula is more complex,
as it must take into account both attacking strength and defensive strength.
Because $n$ dice (proportional to defense) will be rolled,
in order to avoid a compounding impact of defense,
the defense must be divided out of the equation.

The development of the formula is as follows:
The attack value \emph{per creature} is proportional to
the average damage multiplied by the attack bonus factor.
\[
    d_a \left(\frac{10 + b_{a}}{10}\right)^{P}
\]
The total attack value of the unit is then 
\[
    d_a u \left(\frac{10 + b_{a}}{10}\right)^{P}
\]
We divide this by the number of dice to get attack value per die: 
\[
    \frac
        {d_a u \left(\frac{10 + b_{a}}{10}\right)^{P}}
        {n}
\]
But to reduce rounding error we instead use the unrounded formula for $n$: 
\[
    \frac
        {d_a u \left(\frac{10 + b_{a}}{10}\right)^{P}}
        {\frac{h u r^P}{10^P H}}
\]
which simplifies to 
\[
    \frac
        {d_a (10 + b_{a})^P H}
        {h r^P}
\]
This value is proportional to our hit chance.
We scale it to a probability by dividing by $A$:
\[
    \frac
        {d_a (10 + b_{a})^P H}
        {h r^P A}
\]
We now have our true hit chance.
To turn this into a target roll,
we multiply by $S$, round, and subtract from $S$
\[
    S -
    \left\lfloor
        \frac
            {d_a (10 + b_{a})^P H S}
            {h r^P A}
    \right\rceil
\]

\subsection{Examples}\label{sec:examples}

We now demonstrate the use of the formulas through examples.

\subsubsection{Unit of 20 Orcs}

An orc has the following relevant statistics:
\begin{itemize}
    \item $h = 15$ HP (2d8 + 6)
    \item $r = 13$ AC
    \item Javelin Attack:
        \begin{itemize}
            \item $b_j = +5$ to hit
            \item $d_j = 6.5$ damage (d6 + 3)
        \end{itemize}
    \item Greataxe Attack:
        \begin{itemize}
            \item $b_g = +5$ to hit
            \item $d_g = 9.5$ damage (d12 + 3)
        \end{itemize}
\end{itemize}

For 20 orcs, given $A = 100$, $H = 100$, $P = 2$, and $S = 20$
\begin{alignat*}{2}
    n   &=  
        \left\lfloor
            \frac
                {(15) (20) (13)^2}
                {(10)^2 100}
        \right\rceil
            &=  5
    \\
    t_a &=
        20 -
        \left\lfloor
            \frac
                {(6.5) (10 + 5)^2 (100) (20)}
                {(15) (13)^2 (100)}
        \right\rceil
            &=   8
    \\
    t_g &=  
        20 -
        \left\lfloor
            \frac
                {(6.5) (10 + 5)^2 (100) (20)}
                {(15) (13)^2 (100)}
        \right\rceil
            &=  3
\end{alignat*}

A 20-orc unit then has 6 hit points, and its attacks are
$\nicefrac{6}{8}$ and $\nicefrac{6}{3}$.

\subsubsection{Armored vs Unarmored Commoners}

Here we see the effect of armor on the statistics.
A commoner has the following relevant statistics:

\begin{itemize}
    \item $h = 4.5$ HP (1d8)
    \item $r = 10$ AC
    \item Club Attack:
        \begin{itemize}
            \item $b_c = +2$ to hit
            \item $d_c = 2.5$ damage (1d4)
        \end{itemize}
\end{itemize}


For 100 commoners, given $A = 100$, $H = 100$, $P = 2$, $S = 20$

\begin{alignat*}{2}
    n   &=  
        \left\lfloor
            \frac
                {(4.5) (100) (10^2)}
                {(10^2) (100)}
        \right\rceil
            &= 4
    \\
    t_c &=
        20 -
        \left\lfloor
            \frac
                {(2.5) (10 + 2)^2 (100) (20)}
                {(4.5) (10)^2 (100)}
        \right\rceil
            &= 4
\end{alignat*}

Equipped with chain shirts, the commoners' AC rises to 13.
This affects their statistics as follows:

\begin{alignat*}{2}
    n   &=  
        \left\lfloor
            \frac
                {(4.5) (100) (13^2)}
                {(10^2) (100)}
        \right\rceil
            &= 8
    \\
    t_c &=
        20 - 
        \left\lfloor
            \frac
                {(2.5) (10 + 2)^2 (100) (20)}
                {(4.5) (13)^2 (100)}
        \right\rceil
            &= 11
\end{alignat*}

Note that while the armored commoners have more dice,
representing their superior defense,
their larger die size causes each individual die less likely to land a hit.
As the units have the same size and weapons,
their ought to have the same attacking power.
Confirming the validity of the conversion,
the expected damage of each unit is roughly equivalent:
$4(\frac{16}{20}) = 3.2$ for the unarmored commoners,
and $9(\frac{10}{20}) = 3.6$ for the armored commoners.
The discrepancy lies in the choice of parameters;
a larger choice of $S$ and smaller choice of $H$ and $A$
would lead to a less significant difference in damage.

\section{Gameplay}\label{sec:game}
\todo[inline]{move to pre-gameplay: we are still calculating}

\subsection{Scaling Time}
We can calculate the time of round via an additional parameters $F$,
which represents the fraction of a unit that is actually attacking on a given turn.
\[
    s = \frac
        {A}
        {2 H F}
\]
\begin{itemize}
    \item can calculate via assuming only periphery will attack, etc.
\end{itemize}

\todo[inline]{look at \texttt{calc.py} for actual(?) formula}

\todo[inline]{explain why div by 2}
\todo[inline]{explain s}

\subsection{Scaling Space}

Each unit takes up space according to its members:
a unit of 20 medium creatures takes a total of 500 square feet (25 square feet per creature).
When playing on a grid,
some rounding must be done to ensure that each unit conforms to the grid;
i.e., that each unit's tile size is a square number.

We scale the size of units relative to the grid
according to parameter $G$,
which indicates the length of the edge of a grid tile.
A unit's edge length is calculated by the following:
\[
    e = \frac
        {\sqrt{c^2 u}}
        {G}
\]

\subsection{Scaling Speed}
The marching speed $m$ of a unit is determined
by the scaling of time and space
as applied to the creature's speed.
We begin with the creature's speed and divide by 6
to find the distance traveled per second.
\[
    \frac
        {p}
        {6}
\]
We multiply by the number of seconds per round:
\[
    \frac
        {p}
        {6}
    \left(
        \frac
            {A}
            {2 H F}
    \right)
\]
And finally we divide by the number of feet per tile,
simplifying to
\[
    m = \frac
        {p A}
        {12 H F G}
\]

\subsubsection{Minimum Unit Size}

One may select $G$ in such a way that the physically smallest unit
occupies exactly one square.

First, the smallest unit must be found,
which is the unit for which the unscaled size value is smallest:
\[
    \sqrt{c^2 u}
\]
We then select $G$ such that $e = 1$;
i.e., we solve:
\[
    G = \frac
            {\sqrt{c^2 u}}
            {1}
\]
which is easily simplified to
\[
    G = \sqrt{c^2 u}
\]

\subsubsection{Minimum Unit Speed}

One may selection $G$ in such a way that the slowest unit moves one square per turn.

First, the slowest unit must be found,
which is the unit whose creature speed value $p$ is least.

We rearrange the formula for $m$ in order to find the $G$ value:
\[
    G = \frac
            {p A}
            {12 H F m}
\]
Setting $m$ to 1 tile, we have:
\[
    G = \frac
            {p A}
            {12 H F}
\]
\todo[inline]{update according to \texttt{calc.py}}
\todo[inline]{move to ``selecting parameters'' section}

\todo[inline]{Mix the two: take $G = min(G_1, G_2)$}

\todo[inline]{REAL GAMEPLAY SECTION}
\todo[inline]{morale: under half health: DC 10}
\todo[inline]{heroes act specially: zoom in, zoom out}

\missingfigure{diagram of play}

\section{Choosing Parameters}\label{sec:params}

Parameter selection is important for smooth and interesting gameplay.
One should consider a number of options when deciding on parameter values.

\subsection{The $P$ Parameter}

The $P$ parameter is the most nebulous of the parameters,
as it doesn't correspond to any real statistic.
The current recommendation is to assign it a value of 2.

\subsection{The $H$ Parameter}

The $H$ parameter represents the health of the units,
and directly scales their number of dice, which act as hitpoints.

\subsubsection{Minimum Dice}

In order to ensure that each unit has a reasonable number of dice,
it may be desirable to choose a value $n_*$
such that every unit has at least $n_*$ dice.

The unit with the least dice is that with the lowest health,
calculated according to the health factor:
\[
    h u r^P
\]
From the statistics of the least-health unit, the formula for $H$
can easily be derived using the normal $n$ formula.
\[
    n_* =  
        \frac
            {h u r^P}
            {10^P H}
\]
can be rearranged to
\[
    H =  
        \frac
            {h u r^P}
            {10^P n_*}
\]

\todo[inline]{example}

\subsection{The $A$ Parameter}

\todo[inline]{minimum hurdle is bad}
\todo[inline]{try instead: average damage}

The $A$ parameter defines a ``standard'' amount of damage:
the amount of damage a unit must do to reduce an opponent's dice by one.
A low $A$ value can speed up combat,
and may reduce the relative strength of the strongest units;
as hit chance cannot exceed 100\%.
A high $A$ value can slow down combat,
and may leave the weakest units with no chance to hit.
(Units that cannot hit are referred to as \emph{fodder}.)

\subsubsection{Median Damage}

Having already determined the $H$ parameter and
therefore the number of dice of each unit,
may be useful to select $A$ by the average damage of an attack.\footnote{
    Note that the maximum and minimum damage of an attack
    are already determined by the number of dice of each unit,
    as the most damage on unit can do is equal to its health.
}

We begin by identifying the median attack,
according to the attack-defend ratio calculated by the following formula.
\[
    \frac
        {n d_a (10 + b_a)^P}
        {h r^P}
\]
With the median attack in hand,
we can use the unit's statistics to derive an appropriate $A$ value
such that this attack is most likely to deal our target average damage, $x_*$.
Recall the previously introduced hit chance formula:
\[
    p = \frac
        {d_a (10 + b_a)^P H}
        {h r^P A}
\]
We want to choose a value $A$ such that $x_* = n p$:
\[
    x_* = \frac
        {n d_a (10 + b_a)^P, H}
        {h r^P A}
\]
We rearrange to get the formula:
\[
    A = \frac
        {n d_a (10 + b_a)^P, H}
        {h r^P x_*}
\]

\todo[inline]{remove minimum hurdle stuff and clean up}

\subsubsection{Minimum Hurdle}

One may select the minimum hurdle that they would like to see in the game.
For example, one may decide
that the unit with the most accurate attack (per die)
should hit on anything but a 1.
This unit is not necessarily the strongest unit,
but rather the unit whose attacking strength (of its strongest attack)
is most significantly stronger than
its defending strength.

We begin by identifying this unit.
The unit with the highest attack--defend ratio is the unit for which the following formula is maximal.
\[
    \frac
        {d_a (10 + b_a)^P}
        {h r^P}
\]
This is the formula for hit chance, without the common constants.

Taking all units into account and a given $H$ value, $P$ value, and $S$ value,
an $A$ value is selected such that this unit will need to roll $S$ to hit.

We now effectively have a new parameter $t_*$, representing the target roll.
We can then reverse the die formula to derive a formula for $A$:
\[
    t_* =
        S -
        \left\lfloor
            \frac
                {d_a (10 + b_{a})^P H S}
                {h r^P A}
        \right\rceil
\]
\[
    \iff
\]
\[
    S - t_* =
        \left\lfloor
            \frac
                {d_a (10 + b_{a})^P H S}
                {h r^P A}
        \right\rceil
\]
we remove rounding and rearrange to get
\[
    A = 
        \frac
            {d_a (10 + b_{a})^P H S}
            {h r^P (S - t_*)}
\]

\subsubsection{Example: Strongest must roll over 2}

We consider a battle involving 100 unarmored commoners and 20 orcs,
where $P = 2$.
The attack--defend ratio for the commoners is
\[
    \frac
        {2.5 (10 + 2)^2}
        {4.5 (10)^2}
    =
    0.8
\]
whereas for the orcs the ratio (for greataxes) is
\[
    \frac
        {9.5 (10 + 5)^2}
        {15 (13)^2}
    \approx
    0.843
\]

As orcs have a greater ratio, their accuracy will be set at 90\% (rolling over 2).
With $H = 100$, $P = 2$, and $S = 20$,
we solve:
\begin{alignat*}{2}
    A &= 
        \frac
            {(9.5) (10 + 5)^2 (100) (20)}
            {(15) (13)^2 (20 - 2)}
        &{}\approx 93.7
\end{alignat*}

By definition, the orcs' greataxe attack now has a roll hurdle of 2.
Their javelin attack is less likely to deal damage,
and therefore has a higher roll hurdle:
\begin{alignat*}{2}
    t_j &=
        20 -
        \left\lfloor
            \frac
                {(9.5) (10 + 5)^2 (100) (20)}
                {(15) (13)^2 (93.7)}
        \right\rceil
            &= 8
\end{alignat*}

Finally, the commoners have the following roll target
for their club attack.
\begin{alignat*}{2}
    t_c &=
        20 -
        \left\lfloor
            \frac
                {(2.5) (10 + 2)^2 (100) (20)}
                {(4.5) (10)^2 (93.7)}
        \right\rceil
            &= 3
\end{alignat*}

As the size of the respective units and the $P$ and $H$ values
are unchanged from Section~\ref{sec:examples},
the $n$ values calculated for each unit is unchanged.
The roll for the orc unit's greataxe attack is then $\nicefrac{6}{15}$,
whereas for the commoner unit's club attack, it is $\nicefrac{5}{15}$.

\subsubsection{Alternative Hurdle}

Similar calculations may be performed to calculate an $A$ value
such that the maximum hurdle, median hurdle, etc.
is equal to some value.

\subsection{The $S$ Parameter}
The $S$ parameter is generally set to 20 in this document,
as it balances granularity and simplicity,
and is familiar to d20 system users.
It is possible, however, that greater granularity is needed.
This might arise in cases where,
after calculating an appropriate $A$ value,
one or more battle participants needs only to roll 1 to hit,
(i.e., will always hit),
or otherwise if participants whose strength ought to be different
identical statistics.

These issues may be addressed by increasing the $S$ parameter to 100,
for example.
However, they may instead point to an issue that some unit
is so weak as to be in ineffectual and might be disregarded for the battle.

\section{Case Studies}
\todo[inline]{work in progress}

\section{Rules Summary}
\todo[inline]{work in progress}

\section{To Do}
\begin{itemize}
    \item AOE spells: double damage?
    \item Multiattack: roll xN?
    \item Dis/Advantage: roll twice/half?
    \item Morale: roll if took $\geq$ half damage?
\end{itemize}

\end{document}