\subsection{Select general and meta-parameters}

There are two parameters that do not affect scaling,
and generally can be selected before anything else:

\begin{itemize}
    \item $S$ (die size) -- the die to be rolled to determine whether attacks hit
    \item $P$ (accuracy exponent) -- controls the compounding force of accuracy
\end{itemize}

Because the d20 is the typical die for determining whether attacks hit,
it is generally advisable to use $S = 20$.
However, larger dice will lead to greater granularity:
if there is a large disparity in power among units,
$S = 100$ or even greater may be desirable.
It is recommended that $P = 2$.

There are two parameters that may be used to guide selection of the scaling parameters:

\begin{itemize}
    \item $N$ (minimum health) -- the minimum health points for any unit
    \item $D$ (mean damage) -- the average damage done by the average unit on its turn
\end{itemize}

Typically, setting $N = 10$ is a reasonable choice for scaling the battle.
The $D$ parameter is more likely to need adjustment after seeing results,
but $D = 5$ is a reasonable starting value,
as it indicates the average unit could eliminate the weakest unit in two turns.


\todo[inline]{$N$ is $n_*$ in explanations later; $D$ is $d_*$. (But not in the new stuff!)}