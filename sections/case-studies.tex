\section{Case Studies}
\todo[inline]{work in progress}

\subsection{Examples}\label{sec:examples}

We now demonstrate the use of the formulas through examples.

\subsubsection{Unit of 20 Orcs}

An orc has the following relevant statistics:
\begin{itemize}
    \item $h = 15$ HP (2d8 + 6)
    \item $r = 13$ AC
    \item Javelin Attack:
        \begin{itemize}
            \item $b_j = +5$ to hit
            \item $d_j = 6.5$ damage (d6 + 3)
        \end{itemize}
    \item Greataxe Attack:
        \begin{itemize}
            \item $b_g = +5$ to hit
            \item $d_g = 9.5$ damage (d12 + 3)
        \end{itemize}
\end{itemize}

For 20 orcs, given $A = 100$, $H = 100$, $P = 2$, and $S = 20$
\begin{alignat*}{2}
    n   &=
        \left\lfloor
            \frac
                {(15) (20) (13)^2}
                {(10)^2 100}
        \right\rceil
            &=  5
    \\
    t_a &=
        20 -
        \left\lfloor
            \frac
                {(6.5) (10 + 5)^2 (100) (20)}
                {(15) (13)^2 (100)}
        \right\rceil
        + 1
            &=   9
    \\
    t_g &=
        20 -
        \left\lfloor
            \frac
                {(6.5) (10 + 5)^2 (100) (20)}
                {(15) (13)^2 (100)}
        \right\rceil
        + 1
            &=  4
\end{alignat*}

A 20-orc unit then has 6 hit points, and its attacks are
$\nicefrac{6}{8}$ and $\nicefrac{6}{3}$.
\todo[inline]{need to address other stats too}

\subsubsection{Armored vs Unarmored Commoners}

Here we see the effect of armor on the statistics.
A commoner has the following relevant statistics:

\begin{itemize}
    \item $h = 4.5$ HP (1d8)
    \item $r = 10$ AC
    \item Club Attack:
        \begin{itemize}
            \item $b_c = +2$ to hit
            \item $d_c = 2.5$ damage (1d4)
        \end{itemize}
\end{itemize}


For 100 commoners, given $A = 100$, $H = 100$, $P = 2$, $S = 20$

\begin{alignat*}{2}
    n   &=
        \left\lfloor
            \frac
                {(4.5) (100) (10^2)}
                {(10^2) (100)}
        \right\rceil
            &= 4
    \\
    t_c &=
        20 -
        \left\lfloor
            \frac
                {(2.5) (10 + 2)^2 (100) (20)}
                {(4.5) (10)^2 (100)}
        \right\rceil
        + 1
            &= 5
\end{alignat*}

Equipped with chain shirts, the commoners' AC rises to 13.
This affects their statistics as follows:

\begin{alignat*}{2}
    n   &=
        \left\lfloor
            \frac
                {(4.5) (100) (13^2)}
                {(10^2) (100)}
        \right\rceil
            &= 8
    \\
    t_c &=
        20 -
        \left\lfloor
            \frac
                {(2.5) (10 + 2)^2 (100) (20)}
                {(4.5) (13)^2 (100)}
        \right\rceil
        + 1
            &= 12
\end{alignat*}

Note that while the armored commoners have more dice,
representing their superior defense,
their larger die size causes each individual die less likely to land a hit.
As the units have the same size and weapons,
their ought to have the same attacking power.
Confirming the validity of the conversion,
the expected damage of each unit is roughly equivalent:
$4(\frac{16}{20}) = 3.2$ for the unarmored commoners,
and $9(\frac{10}{20}) = 3.6$ for the armored commoners.
The discrepancy lies in the choice of parameters;
a larger choice of $S$ and smaller choice of $H$ and $A$
would lead to a less significant difference in damage.