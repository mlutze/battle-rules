\section{Pre-gameplay}\label{sec:pregame}

Before a battle starts,
calculations are performed based on the participants' D\&D statistics.
There are 4 steps to the pre-gameplay calculation:

\begin{enumerate}
    \item Organize the participants into heterogeneous units.
    \item Gather the relevant statistics for each unit.
    \item Calculate unscaled statistics for each unit.
    \item Select scaling parameters.
    \item Calculate scaled statistics for each unit.
\end{enumerate}

\subsection{Organize}

The first step in preparing a large-scale battle is to organize the participants of the battle into heterogeneous units.
This means that each unit in the battle, which will have one collective turn,
must be made up of creatures with identical attributes and equipment.
Units do not need to be of identical size when compared with each other;
a battle may involve one unit of 50 orcs and another of 100 orcs.
However, the system works best when there are not orders of magnitude
of difference in strength among the units.
For example, many units of ten frogs against a tarrasque
may not be a usable setup for a battle;
instead the frogs should be organized into larger units
that would not be totally insignificant adversaries.
\subsection{Gather Statistics}

The next step is to gather the relevant statistics
for each unit involved in the battle.
The required statistics are summarized below,
each with an associated symbol,
to be used in calculations in Section~\ref{sec:calculate-unscaled} and Section~\ref{sec:calculate-scaled}.
%
\begin{itemize}
    \item $h$ (creature health) -- the maximum hitpoints of a single creature
    \item $r$ (creature AC) -- the AC of a creature
    \item $d_a$ (attack damage) -- the average damage done by attack $a$
    \item $b_a$ (attack bonus) -- the to-hit bonus of attack $a$
    \item $n_a$ (normal range) -- the normal range of attack $a$
    \item $l_a$ (long range) -- the maximum range of attack $a$
    \item $u$ (unit size) -- the number of creatures in the unit
    \item $c$ (creature size) -- the length of a tile that a creature occupies
        (e.g., 5 feet for a medium creature)
    \item $p$ (creature speed) -- the movement speed of a creature
\end{itemize}
%
Some statistics are subscripted, such as $d_a$.
These are per-attack statistics.
A unit with both a sling attack and a sword attack
would have both a $d_\textsf{sling}$ statistic
and a $d_\textsf{sword}$ statistic.
\subsection{Calculate unscaled statistics}\label{sec:calculate-unscaled}

\todo[inline]{actually the order you do this in is kind of complicated...}
\todo[inline]{maybe collecting these unscaled statistics is fine}
\input{sections/pregame/select-parameters.tex}
\subsection{Calculate scaled statistics}\label{sec:calculate-scaled}

Finally, it is possible to calculate the scaled statistics of the units.
Each unit has a set of new statistics,
while one final statistic represents the time of a round of combat.

\paragraph{Per-unit statistics}
\begin{itemize}
    \item $n$ (number of dice) -- the number of dice to use for attacks
    \item $t_a$ (target) -- the number to roll above in order to succeed with attack $a$
    \item $x_a$ (extent) -- the maximum number of tiles away from which attack $a$ can be targeted
    \item $m$ (movement speed) -- the number of tiles a unit moves per turn
    \item $e$ (edge length) -- the number of tiles in a unit's edge length
\end{itemize}

\paragraph{Global statistics}
\begin{itemize}
    \item $s$ (seconds per turn) -- the length of one turn of gameplay in seconds
\end{itemize}

These can be calculated according to the following formulas:

\begin{align*}
\intertext{\textbf{Per-creature values}}
    n   &=
        \left\lfloor
            \frac
                {h u r^P}
                {10^P H}
        \right\rceil \\[2ex]
    t_a &=
        S -
        \left\lfloor
            \frac
                {d_a (10 + b_{a})^P H S}
                {h r^P A}
        \right\rceil \\[2ex]
    x_a &= \left\lceil\frac
        {n_a + l_a}
        {2 G}
        \right\rceil \\[2ex]
    e   &=
        \frac
            {\sqrt{c^2 u}}
            {G} \\[2ex]
\intertext{\textbf{Global values}}
    s   &=
        \frac
            {3 A}
            {H F}
\end{align*}

\subsection{Variables}

We take a single creature's base statistics along with the number of creatures in a unit
in order to calculate that unit's overall strength.


\subsubsection{Output}

There are two output values.\todo{N output values}
The number of dice is an aggregation of a unit's defensive strength and number.
It is used as a set of hitpoints during combat.
The roll target is an aggregation
of a unit's offensive strength and accuracy \emph{per die}.
Regardless of unit size,
two units of creatures with the same statistics
will always have the same roll target
(but may use a different number of dice).
A lower roll target is stronger,
as the goal is to roll above the target,
resulting in reducing the opponent's number of dice by 1.

\paragraph{Per-creature values}
\begin{itemize}
    \item $n$ (number of dice) -- the number of dice to use for attacks
    \item $t_a$ (target) -- the number to roll above in order to succeed with attack $a$
    \item $x_a$ (extent) -- the maximum number of tiles away from which attack $a$ can be targeted
    \item $m$ (movement speed) -- the number of tiles a unit moves per turn
    \item $e$ (edge length) -- the number of tiles in a unit's edge length
\end{itemize}

\paragraph{Global values}
\begin{itemize}
    \item $s$ (seconds per turn) -- the length of one turn of gameplay in seconds
\end{itemize}

\todo[inline]{explain global/per-creature distinction}

The combination of the two statistics is written $\nicefrac{n}{t_a}$
and read ``$n$ over $t_a$''.

\subsection{Calculations}

The output values are calculated according to the following formulas:

\begin{align*}
\intertext{\textbf{Per-creature values}}
    n   &=
        \left\lfloor
            \frac
                {h u r^P}
                {10^P H}
        \right\rceil \\[2ex]
    t_a &=
        S -
        \left\lfloor
            \frac
                {d_a (10 + b_{a})^P H S}
                {h r^P A}
        \right\rceil \\[2ex]
    x_a &= \left\lceil\frac
        {n_a + l_a}
        {2 G}
        \right\rceil \\[2ex]
    e   &=
        \frac
            {\sqrt{c^2 u}}
            {G} \\[2ex]
\intertext{\textbf{Global values}}
    s   &=
        \frac
            {3 A}
            {H F}
\end{align*}


\subsection{Examples}\label{sec:examples}

We now demonstrate the use of the formulas through examples.

\subsubsection{Unit of 20 Orcs}

An orc has the following relevant statistics:
\begin{itemize}
    \item $h = 15$ HP (2d8 + 6)
    \item $r = 13$ AC
    \item Javelin Attack:
        \begin{itemize}
            \item $b_j = +5$ to hit
            \item $d_j = 6.5$ damage (d6 + 3)
        \end{itemize}
    \item Greataxe Attack:
        \begin{itemize}
            \item $b_g = +5$ to hit
            \item $d_g = 9.5$ damage (d12 + 3)
        \end{itemize}
\end{itemize}

For 20 orcs, given $A = 100$, $H = 100$, $P = 2$, and $S = 20$
\begin{alignat*}{2}
    n   &=
        \left\lfloor
            \frac
                {(15) (20) (13)^2}
                {(10)^2 100}
        \right\rceil
            &=  5
    \\
    t_a &=
        20 -
        \left\lfloor
            \frac
                {(6.5) (10 + 5)^2 (100) (20)}
                {(15) (13)^2 (100)}
        \right\rceil
        + 1
            &=   9
    \\
    t_g &=
        20 -
        \left\lfloor
            \frac
                {(6.5) (10 + 5)^2 (100) (20)}
                {(15) (13)^2 (100)}
        \right\rceil
        + 1
            &=  4
\end{alignat*}

A 20-orc unit then has 6 hit points, and its attacks are
$\nicefrac{6}{8}$ and $\nicefrac{6}{3}$.
\todo[inline]{need to address other stats too}

\subsubsection{Armored vs Unarmored Commoners}

Here we see the effect of armor on the statistics.
A commoner has the following relevant statistics:

\begin{itemize}
    \item $h = 4.5$ HP (1d8)
    \item $r = 10$ AC
    \item Club Attack:
        \begin{itemize}
            \item $b_c = +2$ to hit
            \item $d_c = 2.5$ damage (1d4)
        \end{itemize}
\end{itemize}


For 100 commoners, given $A = 100$, $H = 100$, $P = 2$, $S = 20$

\begin{alignat*}{2}
    n   &=
        \left\lfloor
            \frac
                {(4.5) (100) (10^2)}
                {(10^2) (100)}
        \right\rceil
            &= 4
    \\
    t_c &=
        20 -
        \left\lfloor
            \frac
                {(2.5) (10 + 2)^2 (100) (20)}
                {(4.5) (10)^2 (100)}
        \right\rceil
        + 1
            &= 5
\end{alignat*}

Equipped with chain shirts, the commoners' AC rises to 13.
This affects their statistics as follows:

\begin{alignat*}{2}
    n   &=
        \left\lfloor
            \frac
                {(4.5) (100) (13^2)}
                {(10^2) (100)}
        \right\rceil
            &= 8
    \\
    t_c &=
        20 -
        \left\lfloor
            \frac
                {(2.5) (10 + 2)^2 (100) (20)}
                {(4.5) (13)^2 (100)}
        \right\rceil
        + 1
            &= 12
\end{alignat*}

Note that while the armored commoners have more dice,
representing their superior defense,
their larger die size causes each individual die less likely to land a hit.
As the units have the same size and weapons,
their ought to have the same attacking power.
Confirming the validity of the conversion,
the expected damage of each unit is roughly equivalent:
$4(\frac{16}{20}) = 3.2$ for the unarmored commoners,
and $9(\frac{10}{20}) = 3.6$ for the armored commoners.
The discrepancy lies in the choice of parameters;
a larger choice of $S$ and smaller choice of $H$ and $A$
would lead to a less significant difference in damage.
