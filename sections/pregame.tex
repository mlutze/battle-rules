\section{Pre-gameplay}\label{sec:pregame}

Before a battle starts,
calculations are performed based on the participants' D\&D statistics.
There are 4 steps to the pre-gameplay calculation:

\begin{enumerate}
    \item Organize the participants into heterogeneous units.
    \item Gather the relevant statistics for each unit.
    \item Calculate unscaled statistics for each unit.
    \item Select scaling parameters.
    \item Calculate scaled statistics for each unit.
\end{enumerate}

\subsection{Organize}

The first step in preparing a large-scale battle is to organize the participants of the battle into heterogeneous units.
This means that each unit in the battle, which will have one collective turn,
must be made up of creatures with identical attributes and equipment.
Units do not need to be of identical size when compared with each other;
a battle may involve one unit of 50 orcs and another of 100 orcs.
However, the system works best when there are not orders of magnitude
of difference in strength among the units.
For example, many units of ten frogs against a tarrasque
may not be a usable setup for a battle;
instead the frogs should be organized into larger units
that would not be totally insignificant adversaries.
\subsection{Gather Statistics}

The next step is to gather the relevant statistics
for each unit involved in the battle.
The required statistics are summarized below,
each with an associated symbol,
to be used in calculations in Section~\ref{sec:calculate-unscaled} and Section~\ref{sec:calculate-scaled}.
%
\begin{itemize}
    \item $h$ (creature health) -- the maximum hitpoints of a single creature
    \item $r$ (creature AC) -- the AC of a creature
    \item $d_a$ (attack damage) -- the average damage done by attack $a$
    \item $b_a$ (attack bonus) -- the to-hit bonus of attack $a$
    \item $n_a$ (normal range) -- the normal range of attack $a$
    \item $l_a$ (long range) -- the maximum range of attack $a$
    \item $u$ (unit size) -- the number of creatures in the unit
    \item $c$ (creature size) -- the length of a tile that a creature occupies
        (e.g., 5 feet for a medium creature)
    \item $p$ (creature speed) -- the movement speed of a creature
\end{itemize}
%
Some statistics are subscripted, such as $d_a$.
These are per-attack statistics.
A unit with both a sling attack and a sword attack
would have both a $d_\textsf{sling}$ statistic
and a $d_\textsf{sword}$ statistic.
\subsection{Calculate unscaled statistics}\label{sec:calculate-unscaled}

From the general parameters and the input statistics of the units
partaking in the battle,
we next calculate some \emph{unscaled} statistics.
These will be used to help select the parameters in in Section~\ref{sec:params}.
Specifically, the statistics we will calculate are:
%
\begin{itemize}
    \item $n_*$ (health factor) -- a statistic indicating the total defensive power of a unit
    \item $d_*$ (attack factor) -- a statistic indicating the total offensive power of a unit as a proportion of its defense
    \item $e_*$ (space factor) -- a statistic indicating the total space taken by a unit
\end{itemize}
%

These statistics are calculated according to the following formulas:
%
\begin{align*}
    n_* &= hur^P \\
    d_* &= \frac{
        n d_a (10 + b_a)^P
    }{
        h r^P
    } \\
    e_* &= \sqrt{c^2 u}
\end{align*}
%
From these statistics, we calulate the following aggregate statistics:
%
\begin{itemize}
    \item $n_\downarrow$ (minimum health factor) the minimum value of the health factor among all units
    \item $d_\rightarrow$ (median attack factor) the median value of the attack factor among all units
    \item $e_\downarrow$ (minimum space factor) the minimum value of the space factor among all units
    \item $u_\rightarrow$ (median unit number) the median number of creatures in a unit
    \item $p_\downarrow$ (minimum unit speed) the minimum value of the speed of creatures in a unit
\end{itemize}

\subsection{Select parameters}


The parameters are used to scale the battle to an appropriate size.
They control the damage done by attacks,
the health of units in the battle,
and the scaling of the board size,
in turn affecting the duration of the battle.
The parameters are summarized below.

\begin{itemize}
    \item $A$ (attack scalar) -- roughly indicates how much damage is required to reduce a unit's health by one point
    \item $H$ (health scalar) -- indicates how many points of creature health (without armor) are equal to one point of unit health
    \item $P$ (accuracy exponent) -- indicates the compounding force of accuracy
    \item $S$ (the size of the die) -- the die to be rolled to determine if attacks hit
    \item $F$ (fraction attacking) -- the proportion of a unit that is attacking on a turn
    \item $G$ (grid size) -- the number of feet represented by the edge of a grid tile
\end{itemize}

Provided they are ascribed valid values,
the invariant is maintained that if army $A$ is objectively stronger than army $B$,
the resulting statistics of army $A$ will be at least as strong as those of army $B$.

\section{Choosing Parameters}\label{sec:params}

\todo[inline]{meta-parameters}

\subsection{Metaparameters}\label{sec:metaparams}

Because the parameters are fairly abstract,
it better to decide them via metaparameters
that correspond to more concrete statistics.
\todo{introduce n-star and d-star}

\begin{table}
\begin{tabular}{c|c|c|c|c|c}
        &   $n$             &   $t_a$           &   $m$             &   $e$             &   $s$             \\
    \hline
    $A$ &   -               &   $\downarrow$    &   $\uparrow$      &   -               &   $\downarrow$    \\
    $H$ &   $\downarrow$    &   $\uparrow$      &   $\downarrow$    &   -               &   $\uparrow$      \\
    $P$ &   *               &   *               &   -               &   -               &   -               \\
    $S$ &   -               &   $\uparrow$      &   -               &   -               &   -               \\
    $F$ &   -               &   -               &   $\downarrow$    &   -               &   $\downarrow$    \\
    $G$ &   -               &   -               &   $\downarrow$    &   $\downarrow$    &   -
\end{tabular}
\end{table}

\begin{table}
\begin{tabular}{c|c|c|c|c|c|c}
            &   $A$             &   $H$             &   $P$ &   $S$ &   $F$ &   $G$ \\
    \hline
    $n_*$   &   -               &   $\downarrow$    &   -   &   -   &   -   &   -   \\
    $d_*$   &   $\downarrow$    &   -               &   -   &   -   &   -   &   -
\end{tabular}
\end{table}

\begin{table}
\begin{tabular}{c|c|c|c|c|c}
            &   $n$             &   $t_a$           &   $m$             &   $e$             &   $s$             \\
    \hline
    $n_*$   &   $\uparrow$      &   $\downarrow$    &   $\uparrow$      &   -               &   $\downarrow$    \\
    $d_*$   &   -               &   $\uparrow$      &   $\downarrow$    &   -               &   $\uparrow$      \\
\end{tabular}
\end{table}

Parameter selection is important for smooth and interesting gameplay.
One should consider a number of options when deciding on parameter values.

\subsection{The $P$ Parameter}

The $P$ parameter is the most nebulous of the parameters,
as it doesn't correspond to any real statistic.
The current recommendation is to assign it a value of 2.

\subsection{The $H$ Parameter}

The $H$ parameter represents the health of the units,
and directly scales their number of dice, which act as hitpoints.

\subsubsection{Minimum Dice}

In order to ensure that each unit has a reasonable number of dice,
it may be desirable to choose a value $n_*$
such that every unit has at least $n_*$ dice.

The unit with the least dice is that with the lowest health,
calculated according to the health factor:
\[
    h u r^P
\]
From the statistics of the least-health unit, the formula for $H$
can easily be derived using the normal $n$ formula.
\[
    n_* =
        \frac
            {h u r^P}
            {10^P H}
\]
can be rearranged to
\[
    H =
        \frac
            {h u r^P}
            {10^P n_*}
\]

\todo[inline]{example}

\subsection{The $A$ Parameter}

The $A$ parameter defines a ``standard'' amount of damage:
the amount of damage a unit must do to reduce an opponent's dice by one.
A low $A$ value can speed up combat,
and may reduce the relative strength of the strongest units;
as hit chance cannot exceed 100\%.
A high $A$ value can slow down combat,
and may leave the weakest units with no chance to hit.
(Units that cannot hit are referred to as \emph{fodder}.)

\subsubsection{Median Damage}

Having already determined the $H$ parameter and
therefore the number of dice of each unit,
may be useful to select $A$ by the average damage of an attack.\footnote{
    Note that the maximum and minimum damage of an attack
    are already determined by the number of dice of each unit,
    as the most damage on unit can do is equal to its health.
}

We begin by identifying the median attack,
according to the attack-defend ratio calculated by the following formula.
\[
    \frac
        {n d_a (10 + b_a)^P}
        {h r^P}
\]
With the median attack in hand,
we can use the unit's statistics to derive an appropriate $A$ value
such that this attack is most likely to deal our target average damage, $d_*$.
Recall the previously introduced hit chance formula:
\[
    p = \frac
        {d_a (10 + b_a)^P H}
        {h r^P A}
\]
We want to choose a value $A$ such that $d_* = n p$:
\[
    d_* = \frac
        {n d_a (10 + b_a)^P, H}
        {h r^P A}
\]
We rearrange to get the formula:
\[
    A = \frac
        {n d_a (10 + b_a)^P, H}
        {h r^P d_*}
\]

\todo[inline]{pick A by min target $t$}

\subsection{The $S$ Parameter}
The $S$ parameter is generally set to 20 in this document,
as it balances granularity and simplicity,
and is familiar to d20 system users.
It is possible, however, that greater granularity is needed.
This might arise in cases where,
after calculating an appropriate $A$ value,
one or more battle participants needs only to roll 1 to hit,
(i.e., will always hit),
or otherwise if participants whose strength ought to be different
identical statistics.

These issues may be addressed by increasing the $S$ parameter to 100,
for example.
However, they may instead point to an issue that some unit
is so weak as to be in ineffectual and might be disregarded for the battle.

\subsection{The $F$ Parameter}

\subsubsection{Scaling Time}
We can calculate the time of round via the $F$ parameter,
which represents the fraction of a unit that is actually attacking on a given turn.
\[
    s = \frac
        {3 A}
        {H F}
\]
\todo[inline]{can calculate via assuming only periphery will attack, etc.}
\todo[inline]{explain why div by 2}
\todo[inline]{explain why mul by 3 (6 / 2)}
\todo[inline]{explain s}

\subsection{The $G$ Parameter}

\subsubsection{Scaling Space}

Each unit takes up space according to its members:
a unit of 20 medium creatures takes a total of 500 square feet (25 square feet per creature).
When playing on a grid,
some rounding must be done to ensure that each unit conforms to the grid;
i.e., that each unit's tile size is a square number.

We scale the size of units relative to the grid
according to parameter $G$,
which indicates the length of the edge of a grid tile.
A unit's edge length is calculated by the following:
\[
    e = \frac
        {\sqrt{c^2 u}}
        {G}
\]

\subsubsection{Scaling Speed}
The marching speed $m$ of a unit is determined
by the scaling of time and space
as applied to the creature's speed.
We begin with the creature's speed and divide by 6
to find the distance traveled per second.
\[
    \frac
        {p}
        {6}
\]
We multiply by the number of seconds per round:
\[
    \frac
        {p}
        {6}
    \left(
        \frac
            {A}
            {2 H F}
    \right)
\]
And finally we divide by the number of feet per tile,
simplifying to
\[
    m = \frac
        {p A}
        {12 H F G}
\]

\subsubsection{Minimum Unit Size}

One may select $G$ in such a way that the physically smallest unit
occupies exactly one square.

First, the smallest unit must be found,
which is the unit for which the unscaled size value is smallest:
\[
    \sqrt{c^2 u}
\]
We then select $G$ such that $e = 1$;
i.e., we solve:
\[
    G = \frac
            {\sqrt{c^2 u}}
            {1}
\]
which is easily simplified to
\[
    G = \sqrt{c^2 u}
\]

\subsubsection{Minimum Unit Speed}

One may selection $G$ in such a way that the slowest unit moves one square per turn.

First, the slowest unit must be found,
which is the unit whose creature speed value $p$ is least.

We rearrange the formula for $m$ in order to find the $G$ value:
\[
    G = \frac
            {p A}
            {12 H F m}
\]
Setting $m$ to 1 tile, we have:
\[
    G = \frac
            {p A}
            {12 H F}
\]

\subsubsection{Attack Range}

With the size of the grid established, the attack range for the units may be established.
This is done simply by taking the average of the normal and long ranges for an attack,
dividing by the grid size,
and rounding up.

\[
    x_a = \left\lceil\frac
        {n_a + l_a}
        {2 G}
        \right\rceil
\]

\todo[inline]{Mix the two: take $G = min(G_1, G_2)$}
\subsection{Calculate scaled statistics}\label{sec:calculate-scaled}

\subsection{Variables}

We take a single creature's base statistics along with the number of creatures in a unit
in order to calculate that unit's overall strength.


\subsubsection{Output}

There are two output values.\todo{N output values}
The number of dice is an aggregation of a unit's defensive strength and number.
It is used as a set of hitpoints during combat.
The roll target is an aggregation
of a unit's offensive strength and accuracy \emph{per die}.
Regardless of unit size,
two units of creatures with the same statistics
will always have the same roll target
(but may use a different number of dice).
A lower roll target is stronger,
as the goal is to roll above the target,
resulting in reducing the opponent's number of dice by 1.

\paragraph{Per-creature values}
\begin{itemize}
    \item $n$ (number of dice) -- the number of dice to use for attacks
    \item $t_a$ (target) -- the number to roll above in order to succeed with attack $a$
    \item $x_a$ (extent) -- the maximum number of tiles away from which attack $a$ can be targeted
    \item $m$ (movement speed) -- the number of tiles a unit moves per turn
    \item $e$ (edge length) -- the number of tiles in a unit's edge length
\end{itemize}

\paragraph{Global values}
\begin{itemize}
    \item $s$ (seconds per turn) -- the length of one turn of gameplay in seconds
\end{itemize}

\todo[inline]{explain global/per-creature distinction}

The combination of the two statistics is written $\nicefrac{n}{t_a}$
and read ``$n$ over $t_a$''.

\subsection{Calculations}

The output values are calculated according to the following formulas:

\begin{align*}
\intertext{\textbf{Per-creature values}}
    n   &=
        \left\lfloor
            \frac
                {h u r^P}
                {10^P H}
        \right\rceil \\[2ex]
    t_a &=
        S -
        \left\lfloor
            \frac
                {d_a (10 + b_{a})^P H S}
                {h r^P A}
        \right\rceil \\[2ex]
    x_a &= \left\lceil\frac
        {n_a + l_a}
        {2 G}
        \right\rceil \\[2ex]
    e   &=
        \frac
            {\sqrt{c^2 u}}
            {G} \\[2ex]
\intertext{\textbf{Global values}}
    s   &=
        \frac
            {3 A}
            {H F}
\end{align*}


\subsection{Examples}\label{sec:examples}

We now demonstrate the use of the formulas through examples.

\subsubsection{Unit of 20 Orcs}

An orc has the following relevant statistics:
\begin{itemize}
    \item $h = 15$ HP (2d8 + 6)
    \item $r = 13$ AC
    \item Javelin Attack:
        \begin{itemize}
            \item $b_j = +5$ to hit
            \item $d_j = 6.5$ damage (d6 + 3)
        \end{itemize}
    \item Greataxe Attack:
        \begin{itemize}
            \item $b_g = +5$ to hit
            \item $d_g = 9.5$ damage (d12 + 3)
        \end{itemize}
\end{itemize}

For 20 orcs, given $A = 100$, $H = 100$, $P = 2$, and $S = 20$
\begin{alignat*}{2}
    n   &=
        \left\lfloor
            \frac
                {(15) (20) (13)^2}
                {(10)^2 100}
        \right\rceil
            &=  5
    \\
    t_a &=
        20 -
        \left\lfloor
            \frac
                {(6.5) (10 + 5)^2 (100) (20)}
                {(15) (13)^2 (100)}
        \right\rceil
        + 1
            &=   9
    \\
    t_g &=
        20 -
        \left\lfloor
            \frac
                {(6.5) (10 + 5)^2 (100) (20)}
                {(15) (13)^2 (100)}
        \right\rceil
        + 1
            &=  4
\end{alignat*}

A 20-orc unit then has 6 hit points, and its attacks are
$\nicefrac{6}{8}$ and $\nicefrac{6}{3}$.
\todo[inline]{need to address other stats too}

\subsubsection{Armored vs Unarmored Commoners}

Here we see the effect of armor on the statistics.
A commoner has the following relevant statistics:

\begin{itemize}
    \item $h = 4.5$ HP (1d8)
    \item $r = 10$ AC
    \item Club Attack:
        \begin{itemize}
            \item $b_c = +2$ to hit
            \item $d_c = 2.5$ damage (1d4)
        \end{itemize}
\end{itemize}


For 100 commoners, given $A = 100$, $H = 100$, $P = 2$, $S = 20$

\begin{alignat*}{2}
    n   &=
        \left\lfloor
            \frac
                {(4.5) (100) (10^2)}
                {(10^2) (100)}
        \right\rceil
            &= 4
    \\
    t_c &=
        20 -
        \left\lfloor
            \frac
                {(2.5) (10 + 2)^2 (100) (20)}
                {(4.5) (10)^2 (100)}
        \right\rceil
        + 1
            &= 5
\end{alignat*}

Equipped with chain shirts, the commoners' AC rises to 13.
This affects their statistics as follows:

\begin{alignat*}{2}
    n   &=
        \left\lfloor
            \frac
                {(4.5) (100) (13^2)}
                {(10^2) (100)}
        \right\rceil
            &= 8
    \\
    t_c &=
        20 -
        \left\lfloor
            \frac
                {(2.5) (10 + 2)^2 (100) (20)}
                {(4.5) (13)^2 (100)}
        \right\rceil
        + 1
            &= 12
\end{alignat*}

Note that while the armored commoners have more dice,
representing their superior defense,
their larger die size causes each individual die less likely to land a hit.
As the units have the same size and weapons,
their ought to have the same attacking power.
Confirming the validity of the conversion,
the expected damage of each unit is roughly equivalent:
$4(\frac{16}{20}) = 3.2$ for the unarmored commoners,
and $9(\frac{10}{20}) = 3.6$ for the armored commoners.
The discrepancy lies in the choice of parameters;
a larger choice of $S$ and smaller choice of $H$ and $A$
would lead to a less significant difference in damage.
