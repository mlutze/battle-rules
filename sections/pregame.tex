\section{Pre-gameplay}\label{sec:pregame}

Before a battle starts,
calculations are performed based on the participants' D\&D statistics.

\subsection{Parameters}

The parameters are used to scale the battle to an appropriate size.
They control the likelihood of an attack to hit,
and number of dice rolled,
in turn affecting the duration of the battle.

\begin{itemize}
    \item $A$ (attack scalar) -- roughly indicates how much damage is required to reduce a unit's health by one point
    \item $H$ (health scalar) -- indicates how many points of creature health (without armor) are equal to one point of unit health
    \item $P$ (accuracy exponent) -- indicates the compounding force of accuracy
    \item $S$ (the size of the die) -- the die to be rolled to determine if attacks hit
    \item $F$ (fraction attacking) -- the proportion of a unit that is attacking on a turn
    \item $G$ (grid size) -- the number of feet represented by the edge of a grid tile
\end{itemize}

Provided they are ascribed valid values,
the invariant is maintained that if army $A$ is objectively stronger than army $B$,
the resulting statistics of army $A$ will be at least as strong as those of army $B$.

The valid ranges for the parameters are the following:
\begin{align*}
    0~<~&A \\
    0~<~&H \\
    0~\leq~&P \\
    0~<~&S \\
    0~<~&G \\
    0~<~&F~\leq~1
\end{align*}
Section~\ref{sec:params} contains further guidance on selecting values for the parameters.

\subsection{Variables}

We take a single creature's base statistics along with the number of creatures in a unit
in order to calculate that unit's overall strength.

\subsubsection{Input}

As input, we use a creature's health and AC for its defense,
and the damage and attack bonus for each of its attacks,
as well as the unit size.

\begin{itemize}
    \item $h$ (creature health) -- the maximum hitpoints of a single creature
    \item $r$ (creature AC) -- the AC of a creature
    \item $d_a$ (attack damage) -- the average damage done by attack $a$
    \item $b_a$ (attack bonus) -- the to-hit bonus of attack $a$
    \item $n_a$ (normal range) -- the normal range of attack $a$
    \item $l_a$ (long range) -- the maximum range of attack $a$
    \item $u$ (unit size) -- the number of creatures in a unit
    \item $c$ (creature size) -- the length of a tile that a creature occupies
        (e.g., 5 feet for a medium creature)
    \item $p$ (creature speed) -- the movement speed of a creature
\end{itemize}

\subsubsection{Output}

There are two output values.
The number of dice is an aggregation of a unit's defensive strength and number.
It is used as a set of hitpoints during combat.
The roll target is an aggregation
of a unit's offensive strength and accuracy \emph{per die}.
Regardless of unit size,
two units of creatures with the same statistics
will always have the same roll target
(but may use a different number of dice).
A lower roll target is stronger,
as the goal is to roll above the target,
resulting in reducing the opponent's number of dice by 1.

\paragraph{Per-creature values}
\begin{itemize}
    \item $n$ (number of dice) -- the number of dice to use for attacks
    \item $t_a$ (target) -- the number to roll above in order to succeed with attack $a$
    \item $x_a$ (extent) -- the maximum number of tiles away from which attack $a$ can be targeted
    \item $m$ (movement speed) -- the number of tiles a unit moves per turn
    \item $e$ (edge length) -- the number of tiles in a unit's edge length
\end{itemize}

\paragraph{Global values}
\begin{itemize}
    \item $s$ (seconds per turn) -- the length of one turn of gameplay in seconds
\end{itemize}

\todo[inline]{explain global/per-creature distinction}

The combination of the two statistics is written $\nicefrac{n}{t_a}$
and read ``$n$ over $t_a$''.

\subsection{Calculations}

The output values are calculated according to the following formulas:

\begin{align*}
    n   &=
        \left\lfloor
            \frac
                {h u r^P}
                {10^P H}
        \right\rceil \\[2ex]
    t_a &=
        S -
        \left\lfloor
            \frac
                {d_a (10 + b_{a})^P H S}
                {h r^P A}
        \right\rceil \\[2ex]
    x_a &= \left\lceil\frac
        {n_a + l_a}
        {2 G}
        \right\rceil \\[2ex]
    e   &=
        \frac
            {\sqrt{c^2 u}}
            {G} \\[2ex]
    s   &=
        \frac
            {3 A}
            {H F}
\end{align*}

\subsubsection{Accuracy}

D\&D's concept of accuracy is a source of complications in the formulas,
and must be addressed specifically.

We begin by observing that in standard D\&D rules,
an increase by a factor of $x$ in an attack's accuracy
can be roughly equivalently represented by a decrease in a defender's
health by a factor of $x$.
For example, if an attack normally has a 25\% chance of hitting a defender with 2 hitpoints,
we might reasonably expect the defender to last 8 rounds.\footnote{
    This approximation follows intuition,
    but is not statistically correct with respect to expected values.
    It is considered sufficient for this draft
    but may be revisited in a future version.
}
If we raise the accuracy to 50\%,
the defender would need around 4 hitpoints to last 8 rounds.

Accuracy of attacks in standard D\&D is not derived from one statistic,
but from two: the attacker's attack bonus and the defender's AC.
Modeling this directly would present a problem for the transformation,
as it would necessitate an additional statistic and comparison during gameplay.
Instead, we try to bundle the effect of AC and health into $n$,
and the effect of attack bonus into $t_a$.

It is not possible to replicate the effect of accuracy fully, however.
To demonstrate, we turn to an example of a commoner (+2 attack bonus)
attacking a tarrasque (25 AC).
Barring critical hits, the commoner will never be able to hit the tarrasque,
as the maximum roll of 22 is less than the tarrasque's AC.
The result is that the $n$ value for a tarrasque unit must be infinite,
or that the $t_a$ value for a commoner unit must be at least 20.

Clearly, neither of these options is practical.
Instead of modeling the asymptotic effect of AC and attack bonus directly,
we use power function to approximate it.
AC is converted into a defense factor:
\[
    \left(\frac{r}{10}\right)^P
\]
AC is first divided by 10,
as 10 is the standard AC without bonuses or penalties,
then raised to a power $P$ which represents the multiplicative power of AC.

The factor for attack bonus is derived similarly,
but 10 added, establishing 0 as the baseline attack bonus:
\[
    \left(\frac{10 + b_a}{10}\right)^P
\]

\subsubsection{Deriving the Formulas}

The $n$ formula is relatively simple:
It is the total health of the unit,
multiplied by a factor representing the value of armor,
divided by the $H$ parameter:
\[
    \frac
        {h u (\frac{r}{10})^P}
        {H}
\]
We round and simplify to get our value for $n$.

The $t$ formula is more complex,
as it must take into account both attacking strength and defensive strength.
Because $n$ dice (proportional to defense) will be rolled,
in order to avoid a compounding impact of defense,
the defense must be divided out of the equation.

The development of the formula is as follows:
The attack value \emph{per creature} is proportional to
the average damage multiplied by the attack bonus factor.
\[
    d_a \left(\frac{10 + b_{a}}{10}\right)^{P}
\]
The total attack value of the unit is then
\[
    d_a u \left(\frac{10 + b_{a}}{10}\right)^{P}
\]
We divide this by the number of dice to get attack value per die:
\[
    \frac
        {d_a u \left(\frac{10 + b_{a}}{10}\right)^{P}}
        {n}
\]
But to reduce rounding error we instead use the unrounded formula for $n$:
\[
    \frac
        {d_a u \left(\frac{10 + b_{a}}{10}\right)^{P}}
        {\frac{h u r^P}{10^P H}}
\]
which simplifies to
\[
    \frac
        {d_a (10 + b_{a})^P H}
        {h r^P}
\]
This value is proportional to our hit chance.
We scale it to a probability by dividing by $A$:
\[
    \frac
        {d_a (10 + b_{a})^P H}
        {h r^P A}
\]
We now have our true hit chance.
To turn this into a target roll,
we multiply by $S$, round, subtract from $S$ and add 1.
\[
    S -
    \left\lfloor
        \frac
            {d_a (10 + b_{a})^P H S}
            {h r^P A}
    \right\rceil
    + 1
\]

\subsection{Examples}\label{sec:examples}

We now demonstrate the use of the formulas through examples.

\subsubsection{Unit of 20 Orcs}

An orc has the following relevant statistics:
\begin{itemize}
    \item $h = 15$ HP (2d8 + 6)
    \item $r = 13$ AC
    \item Javelin Attack:
        \begin{itemize}
            \item $b_j = +5$ to hit
            \item $d_j = 6.5$ damage (d6 + 3)
        \end{itemize}
    \item Greataxe Attack:
        \begin{itemize}
            \item $b_g = +5$ to hit
            \item $d_g = 9.5$ damage (d12 + 3)
        \end{itemize}
\end{itemize}

For 20 orcs, given $A = 100$, $H = 100$, $P = 2$, and $S = 20$
\begin{alignat*}{2}
    n   &=
        \left\lfloor
            \frac
                {(15) (20) (13)^2}
                {(10)^2 100}
        \right\rceil
            &=  5
    \\
    t_a &=
        20 -
        \left\lfloor
            \frac
                {(6.5) (10 + 5)^2 (100) (20)}
                {(15) (13)^2 (100)}
        \right\rceil
        + 1
            &=   9
    \\
    t_g &=
        20 -
        \left\lfloor
            \frac
                {(6.5) (10 + 5)^2 (100) (20)}
                {(15) (13)^2 (100)}
        \right\rceil
        + 1
            &=  4
\end{alignat*}

A 20-orc unit then has 6 hit points, and its attacks are
$\nicefrac{6}{8}$ and $\nicefrac{6}{3}$.
\todo[inline]{need to address other stats too}

\subsubsection{Armored vs Unarmored Commoners}

Here we see the effect of armor on the statistics.
A commoner has the following relevant statistics:

\begin{itemize}
    \item $h = 4.5$ HP (1d8)
    \item $r = 10$ AC
    \item Club Attack:
        \begin{itemize}
            \item $b_c = +2$ to hit
            \item $d_c = 2.5$ damage (1d4)
        \end{itemize}
\end{itemize}


For 100 commoners, given $A = 100$, $H = 100$, $P = 2$, $S = 20$

\begin{alignat*}{2}
    n   &=
        \left\lfloor
            \frac
                {(4.5) (100) (10^2)}
                {(10^2) (100)}
        \right\rceil
            &= 4
    \\
    t_c &=
        20 -
        \left\lfloor
            \frac
                {(2.5) (10 + 2)^2 (100) (20)}
                {(4.5) (10)^2 (100)}
        \right\rceil
        + 1
            &= 5
\end{alignat*}

Equipped with chain shirts, the commoners' AC rises to 13.
This affects their statistics as follows:

\begin{alignat*}{2}
    n   &=
        \left\lfloor
            \frac
                {(4.5) (100) (13^2)}
                {(10^2) (100)}
        \right\rceil
            &= 8
    \\
    t_c &=
        20 -
        \left\lfloor
            \frac
                {(2.5) (10 + 2)^2 (100) (20)}
                {(4.5) (13)^2 (100)}
        \right\rceil
        + 1
            &= 12
\end{alignat*}

Note that while the armored commoners have more dice,
representing their superior defense,
their larger die size causes each individual die less likely to land a hit.
As the units have the same size and weapons,
their ought to have the same attacking power.
Confirming the validity of the conversion,
the expected damage of each unit is roughly equivalent:
$4(\frac{16}{20}) = 3.2$ for the unarmored commoners,
and $9(\frac{10}{20}) = 3.6$ for the armored commoners.
The discrepancy lies in the choice of parameters;
a larger choice of $S$ and smaller choice of $H$ and $A$
would lead to a less significant difference in damage.
