\section{Derivation}

\subsubsection{Accuracy}

D\&D's concept of accuracy is a source of complications in the formulas,
and must be addressed specifically.

We begin by observing that in standard D\&D rules,
an increase by a factor of $x$ in an attack's accuracy
can be roughly equivalently represented by a decrease in a defender's
health by a factor of $x$.
For example, if an attack normally has a 25\% chance of hitting a defender with 2 hitpoints,
we might reasonably expect the defender to last 8 rounds.\footnote{
    This approximation follows intuition,
    but is not statistically correct with respect to expected values.
    It is considered sufficient for this draft
    but may be revisited in a future version.
}
If we raise the accuracy to 50\%,
the defender would need around 4 hitpoints to last 8 rounds.

Accuracy of attacks in standard D\&D is not derived from one statistic,
but from two: the attacker's attack bonus and the defender's AC.
Modeling this directly would present a problem for the transformation,
as it would necessitate an additional statistic and comparison during gameplay.
Instead, we try to bundle the effect of AC and health into $n$,
and the effect of attack bonus into $t_a$.

It is not possible to replicate the effect of accuracy fully, however.
To demonstrate, we turn to an example of a commoner (+2 attack bonus)
attacking a tarrasque (25 AC).
Barring critical hits, the commoner will never be able to hit the tarrasque,
as the maximum roll of 22 is less than the tarrasque's AC.
The result is that the $n$ value for a tarrasque unit must be infinite,
or that the $t_a$ value for a commoner unit must be at least 20.

Clearly, neither of these options is practical.
Instead of modeling the asymptotic effect of AC and attack bonus directly,
we use power function to approximate it.
AC is converted into a defense factor:
\[
    \left(\frac{r}{10}\right)^P
\]
AC is first divided by 10,
as 10 is the standard AC without bonuses or penalties,
then raised to a power $P$ which represents the multiplicative power of AC.

The factor for attack bonus is derived similarly,
but 10 added, establishing 0 as the baseline attack bonus:
\[
    \left(\frac{10 + b_a}{10}\right)^P
\]

\subsubsection{Deriving the Formulas}

The $n$ formula is relatively simple:
It is the total health of the unit,
multiplied by a factor representing the value of armor,
divided by the $H$ parameter:
\[
    \frac
        {h u (\frac{r}{10})^P}
        {H}
\]
We round and simplify to get our value for $n$.

The $t$ formula is more complex,
as it must take into account both attacking strength and defensive strength.
Because $n$ dice (proportional to defense) will be rolled,
in order to avoid a compounding impact of defense,
the defense must be divided out of the equation.

The development of the formula is as follows:
The attack value \emph{per creature} is proportional to
the average damage multiplied by the attack bonus factor.
\[
    d_a \left(\frac{10 + b_{a}}{10}\right)^{P}
\]
The total attack value of the unit is then
\[
    d_a u \left(\frac{10 + b_{a}}{10}\right)^{P}
\]
We divide this by the number of dice to get attack value per die:
\[
    \frac
        {d_a u \left(\frac{10 + b_{a}}{10}\right)^{P}}
        {n}
\]
But to reduce rounding error we instead use the unrounded formula for $n$:
\[
    \frac
        {d_a u \left(\frac{10 + b_{a}}{10}\right)^{P}}
        {\frac{h u r^P}{10^P H}}
\]
which simplifies to
\[
    \frac
        {d_a (10 + b_{a})^P H}
        {h r^P}
\]
This value is proportional to our hit chance.
We scale it to a probability by dividing by $A$:
\[
    \frac
        {d_a (10 + b_{a})^P H}
        {h r^P A}
\]
We now have our true hit chance.
To turn this into a target roll,
we multiply by $S$, round, subtract from $S$ and add 1.
\[
    S -
    \left\lfloor
        \frac
            {d_a (10 + b_{a})^P H S}
            {h r^P A}
    \right\rceil
    + 1
\]